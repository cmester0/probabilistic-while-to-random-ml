% -------------------------------------------------------------
% Document setup
% -------------------------------------------------------------
\documentclass[11pt, leqno]{article}
\usepackage{graphicx}
\usepackage[left=30mm,right=30mm]{geometry}
\usepackage[utf8]{inputenc}
\usepackage[english]{babel}
% -------------------------------------------------------------
% Notation aids 
% -------------------------------------------------------------
\usepackage{amsmath} % American Mathematical Society math-notations
\usepackage{amssymb} % American Mathematical Society math-symbols
\usepackage{amsthm}  % For good looking definitions and theorems
\usepackage{stmaryrd} % For interpretation bracket (and maybe other stuff)
\usepackage{listings}

% -------------------------------------------------------------
% For convenient development
% -------------------------------------------------------------
\usepackage{todonotes}

\lstdefinelanguage{Coq}{ 
%
% Anything betweeen $ becomes LaTeX math mode
mathescape=true,
%
% Comments may or not include Latex commands
texcl=false, 
%
% Vernacular commands
morekeywords=[1]{Section, Module, End, Require, Import, Export,
  Variable, Variables, Parameter, Parameters, Axiom, Hypothesis,
  Hypotheses, Notation, Local, Tactic, Reserved, Scope, Open, Close,
  Bind, Delimit, Definition, Let, Ltac, Fixpoint, CoFixpoint, Add,
  Morphism, Relation, Implicit, Arguments, Unset, Contextual,
  Strict, Prenex, Implicits, Inductive, CoInductive, Record,
  Structure, Canonical, Coercion, Context, Class, Global, Instance,
  Program, Infix, Theorem, Lemma, Corollary, Proposition, Fact,
  Remark, Example, Proof, Goal, Save, Qed, Defined, Hint, Resolve,
  Rewrite, View, Search, Show, Print, Printing, All, Eval, Check,
  Projections, inside, outside, Def},
%
% Gallina
morekeywords=[2]{forall, exists, exists2, fun, fix, cofix, struct,
  match, with, end, as, in, return, let, if, is, then, else, for, of,
  nosimpl, when},
%
% Sorts
morekeywords=[3]{Type, Prop, Set, true, false, option},
%
% Various tactics, some are std Coq subsumed by ssr, for the manual purpose
morekeywords=[4]{pose, set, move, case, elim, apply, clear, hnf,
  intro, intros, generalize, rename, pattern, after, destruct,
  induction, using, refine, inversion, injection, rewrite, congr,
  unlock, compute, ring, field, fourier, replace, fold, unfold,
  change, cutrewrite, simpl, have, suff, wlog, suffices, without,
  loss, nat_norm, assert, cut, trivial, revert, bool_congr, nat_congr,
  symmetry, transitivity, auto, split, left, right, autorewrite},
%
% Terminators
morekeywords=[5]{by, done, exact, reflexivity, tauto, romega, omega,
  assumption, solve, contradiction, discriminate},
%
% Control
morekeywords=[6]{do, last, first, try, idtac, repeat},
% % Custom
morekeywords=[7]{Rml, sRml},
morekeywords=[8]{Var, Const, Let_stm, Fun_stm, If_stm, App_stm, Let_rec},
morekeywords=[9]{sVar, sConst, sFun, sIf, sApp, sFix},
% Comments delimiters, we do turn this off for the manual
morecomment=[s]{(*}{*)},
%
% Spaces are not displayed as a special character
showstringspaces=false,
%
% String delimiters
morestring=[b]",
morestring=[d]’,
%
% Size of tabulations
tabsize=3,
%
% Enables ASCII chars 128 to 255
extendedchars=false,
%
% Case sensitivity
sensitive=true,
%
% Automatic breaking of long lines
breaklines=false,
%
% Default style fors listings
basicstyle=\small,
%
% Position of captions is bottom
captionpos=b,
%
% flexible columns
columns=[l]flexible,
%
% Style for (listings') identifiers
identifierstyle={\ttfamily\color{black}},
% Style for declaration keywords
keywordstyle=[1]{\ttfamily\color{violet}},
% Style for gallina keywords
keywordstyle=[2]{\ttfamily\color{green}},
% Style for sorts keywords
keywordstyle=[3]{\ttfamily\color{blue}},
% Style for tactics keywords
keywordstyle=[4]{\ttfamily\color{blue}},
% Style for terminators keywords
keywordstyle=[5]{\ttfamily\color{red}},
%Style for iterators
%keywordstyle=[6]{\ttfamily\color{dkpink}},
keywordstyle=[7]{\ttfamily\color{orange}},
keywordstyle=[8]{\ttfamily\color{red}},
keywordstyle=[9]{\ttfamily\color{magenta}},
% Style for strings
stringstyle=\ttfamily,
% Style for comments
commentstyle={\ttfamily\color{purple}},
%
%moredelim=**[is][\ttfamily\color{red}]{/&}{&/},
literate=
    {\\forall}{{\color{dkgreen}{$\forall\;$}}}1
    {\\exists}{{$\exists\;$}}1
    {<-}{{$\leftarrow\;$}}1
    {=>}{{$\Rightarrow\;$}}1
    {==}{{\code{==}\;}}1
    {==>}{{\code{==>}\;}}1
%    {:>}{{\code{:>}\;}}1
    {->}{{$\rightarrow\;$}}1
    {<->}{{$\leftrightarrow\;$}}1
    {<==}{{$\leq\;$}}1
    {\#}{{$^\star$}}1 
    {\\o}{{$\circ\;$}}1 
    {\@}{{$\cdot$}}1 
    {\/\\}{{$\wedge\;$}}1
    {\\\/}{{$\vee\;$}}1
    {++}{{\code{++}}}1
    {~}{{\ }}1
    {nat}{{$\mathbb{N}$}}1
    {forall}{{$\forall$}}1
    {\@\@}{{$@$}}1
    {\\mapsto}{{$\mapsto\;$}}1
    {\\hline}{{\rule{\linewidth}{0.5pt}}}1
%
}[keywords,comments,strings]

\lstnewenvironment{coq}{\lstset{language=Coq}}{}

% pour inliner dans le texte
\def\coqe{\lstinline[language=Coq, basicstyle=\small]}
% pour inliner dans les tableaux / displaymath...
\def\coqes{\lstinline[language=Coq, basicstyle=\scriptsize]}

% \lstdefinelanguage{rml}{
% }

% \lstdefinelanguage{pwhile}{
% }
\lstdefinelanguage{pwhile}{
}
% -------------------------------------------------------------
% Prettyfying
% -------------------------------------------------------------
\usepackage[hidelinks]{hyperref}

\usepackage{xcolor}
% -------------------------------------------------------------

\author{Lasse Letager Hansen %201508114
  \\ Kira Kutscher %201509720
}
\date{\today}
\title{Technical report}

\newcommand{\set}[1]{\{#1\}}

\newcommand{\Rml}{\textcolor{orange}{\mathtt{Rml}}}
\newcommand{\sRml}{\textcolor{orange}{\mathtt{sRml}}}
\newcommand{\Type}{\textcolor{blue}{\mathtt{Type}}}
\newcommand{\wellformed}{\texttt{well\_formed}}
\newcommand{\valid}{\texttt{rml\_valid\_type}}
\newcommand{\bind}{\textcolor{magenta}{>\!\!>\!\!=}}

\theoremstyle{definition}
\newtheorem{thm}{Theorem}
\newtheorem{defn}[thm]{Definition}

\begin{document}

% -------------------------------------------------------------

\newcommand\rml{$\mathcal{R}$\texttt{ml} } % for a pretty version of "Rml"
\newcommand\M{\texttt{M}} % for the M representing monads

% -------------------------------------------------------------

\maketitle

\tableofcontents
\newpage

% ------------------------------------------------------------- 

\section{Introduction}
% Goal: 1-2 pages
% I've written an introduction already, but I don't know if it still fits. We should
% probably write the rest of the report first before adding this section

% here we should write something about the motivation of the project and give an
% overview of what Coq/formal verification is, some very basic things about
% cryptography, and how they relate.


\section{Theory and existing frameworks}
% Goal: 6-12 pages

\subsection{\rml}
We have two representations of Rml, continuations and distributions. Both build on a monad, for ease of use.
\\ \\
The data structure used to represent Rml terms is as follows:
\begin{center}
  \begin{minipage}{0.7\linewidth}
    \begin{lstlisting}[language=coq]
Inductive Rml :=
| Var : (nat * Type) -> Rml
| Const : forall (A : Type), A -> Rml
| Let_stm : (nat * Type) -> Rml -> Rml -> Rml
| Fun_stm : Type -> (nat * Type) -> Rml -> Rml
| If_stm : Rml -> Rml -> Rml -> Rml
| App_stm : Type -> Rml -> Rml -> Rml
| Let_rec : Type -> Type -> nat -> nat -> Rml -> Rml -> Rml.
    \end{lstlisting}
  \end{minipage}
\end{center}
We use all cog types, as possible types of Rml expressions, since there are no real restrictions on the types. We encode variables, as a type and a natural number, so two variables are the same only if they have the same number and refer to the same type.
\\ \\
We have defined a relation \wellformed, that checks that no variables are escaping the scope of an Rml program, that is there is always a binding for an expression of type \(\mathtt{Var}~p\). We furthermore define a relation \valid, which checks that a given Rml expression can be typed under a given type. We have shown that if a Rml program is valid then it is well formed. We have then constructed a simplified form of Rml called sRml (for simple Rml), to make it easier to reason about and evaluate expressions, with the following data structure:
\begin{center}
  \begin{minipage}{0.7\linewidth}
    \begin{lstlisting}[language=coq]
Inductive sRml {A : Type} :=
| sVar : nat -> sRml
| sConst : A -> sRml
| sFun : forall C (p : nat * Type), A = (p.2 -> C) -> @sRml C -> sRml
| sIf : @sRml bool -> sRml -> sRml -> sRml
| sApp : forall T, @sRml (T -> A) -> @sRml T -> sRml
| sFix : forall B (nf nx : nat), @sRml (B -> A) -> @sRml B -> sRml.
    \end{lstlisting}
  \end{minipage}
\end{center}
That is Rml where we remove expressions with variables, from \texttt{let\_stm} statements (not \texttt{let\_rec} statements). We then show that given a valid typing of an Rml expression, we can simplify that expression, and maintain the valid typing (under the same type). With this we can make an interpreter from an interpreter of sRml, which can be constructed as (for continuations). We have a similar function for Rml, using the posibility distributions as interpretations. We see similar patterns arising, since both interpretations are monadic.

\subsection{\textsc{EasyCrypt} and \texttt{pwhile}}
% Here we should give a quick description of EasyCrypt and present pwhile
\textsc{EasyCrypt} is a framework that has been developed in order to help in the
construction of machine-checkable proofs about cryptographic constructions and
protocols. %\cite{ https://www.easycrypt.info/documentation/refman.pdf}
A standard approach to this kind of proofs is based on so-called games; in
\textsc{EasyCrypt} cryptographic algorithms as well as games are modelled as
\textit{modules} consisting of procedures written in a simple imperative language
called \texttt{pwhile}. The \texttt{p} in \texttt{pwhile} stands for
``probabilistic'', so in total the name refers to a probabilistic extension of the
well-known minimalistic \texttt{while} language.

We will in this section give an overview of the language as well as its
interpretation in Coq, which is due to a development by Pierre-Yves Strub
\footnote{https://github.com/strub/xhl}. We will not concern ourselves with the
module system of \textsc{EasyCrypt} since the focus of the present development is
on probabilistic languages and their interpretation rather than their use.\\
\\
\texttt{pwhile} consists of the following expressions and commands: 
\begin{align*}
  exp ::=~& x ~\vert ~ const ~\vert ~ \texttt{prp } pred~mem ~\vert ~ e_1\ e_2\\
  cmd ::=~& \texttt{abort} ~\vert ~ \texttt{skip} ~\vert ~ x := e ~\vert ~ x\ \$= e\\
  \vert ~ & \texttt{if } b \texttt{ then } c_1 \texttt{ else } c_2 ~\vert ~
            \texttt{while } b \texttt{ do } c ~\vert ~ c_1 ; c_2
\end{align*}

\todo[inline]{Explain what prp is all about}

The result of interpreting a program in \texttt{pwhile} is a distribution over
memories.

\todo{Is it enough to describe what the semantics is, or should I extract the formal
  semantics from the xhl development? }


\subsection{Complete partial orders}
% Describe cpos, omega-cpos, and dcpos
% partial order: a set with an ordering relation, but not every two elements have to
% be comparable

A partially ordered set (poset) is a set with an associated binary ordering relation
$\leq$ which is both reflexive and transitive. The order is partial when the ordering
relation is not defined on every pair of elements in the set.

There exist a number of different completeness properties that a poset can have.
We will here have a look at $\omega$-complete partial orders, which we will use in
order to interpret general recursion and randomised programs. 

\begin{defn}
  \textit{$\omega$-complete partial order ($\omega$-cpo)}\\
  An $\omega$-cpo is a partially ordered set that, additionally, has a distinct least
  element and where there exist least upper bounds on all monotonic sequences. 
\end{defn}


\subsubsection{Recursive definitions as fixed point iterations}
% Describe the fixpoint interpretation of recursive/iterative definitions

Before using $\omega$-cpos to interpret recursion, let us first have a look at some
interesting things that our definition entails.

A monotonic sequence on an $\omega$-cpo $X$ can be viewed as a monotonic function
$f~:~\mathbb{N} \xrightarrow{m} X$ where $f(n)$ is the $n$th element of the sequence (or the
least upper bound of the sequence, if $n$ is larger than the length of the
sequence).\\
\\
% Standard definition of fixed points
There is a standard way of defining fixed point iterations on an $\omega$-cpo: % \cite{alea}

Consider an operator $F~:~X \xrightarrow{m} X$ on some $\omega$-cpo $X$; with this we
define the monotonic sequence $F_i \mapsto \underbrace{ F(F(\dots F}_{i \text{
    times}} (0_X) \dots))$ of repeated application of $F$ to the least element of $X$.
By our choice of $F$ and the definition of $\omega$-cpos, it is clear that there has
to exist a least upper bound on $F_i$. This least upper bound is the fixed point of
$F$ and it will hold that $\texttt{fix } F = F(\texttt{fix }F)$ if $F$ is
continuous. 
% @Bas: Where do we need non-continuous functions?

% How it extends to function spaces
For an $\omega$-cpo with underlying set $B$ we can also define an $\omega$-cpo on
functions from any set $A$ whose co-domain is $B$.

To reiterate the definition, let us think of what we need for an $\omega$-cpo. We
need an ordering relation, a least element, and a least upper bound operation. Those
can be defined as follows:
\begin{align*}
  f\leq_{A \to B} g \Leftrightarrow \forall x: f(x) \leq_B g(x) & ~~~\textit{(pointwise order)}\\
  0_{A\to B} := f(x) = 0_B & ~~~\textit{(least element)}\\
  \texttt{lub}_{A\to B} f_n := g(x) = \texttt{lub}_B(f_n(x)) & ~~~\textit{(least upper
                                                               bound operation)}
\end{align*}


% How this can be used to interpret recursion
The result of an interpretation of programs in the language of discourse will be in
an $\omega$-cpo, so according to the above discussion functions will have an
$\omega$-cpo structure as well. Together with the above definition of fixed points we
can use this structure to interpret general recursive definitions. 

We define a functional, $F$, taking as input a function and ``adding a step to it''.
Let us look at the example of the factorial function $f(n) = n!$. %\cite{haskell}
The recursive definition is well known:
$$fac(n) := \texttt{ if } n = 0 \texttt{ then }1\texttt{ else } n\cdot fac(n-1)$$

For the interpretation of this definition, we want to define $F~:~(\mathbb{N} \to
\mathbb{N}) \to (\mathbb{N} \to \mathbb{N})$ in such a way that its fixed point is
the same as the above recursive definition. We choose
$$F(F_i(n)):=\texttt{ if }n=0\texttt{ then }1\texttt{ else }n\cdot F_i(n-1)$$
Where $F_0(n)$ is $0_{\mathbb{N} \to \mathbb{N}}$ (the function that takes a natural
number and returns 0), by the above definition of the least element in the
$\omega$-cpo defined on a function space. By repeated application of $F$ the function
will slowly approach the real factorial function, which is the fixed point of $F$.
The beginning of the iteration will be

\begin{align*}
  F_1(n) = F(F_0(n)) & = \begin{cases}
                           1~~~\text{if }n\text{ is 0}\\
                           0~~~\text{otherwise}
                         \end{cases}
  \\
  F_2(n) = F(F(F_0(n))) & = \begin{cases}
                             1~~~\text{if }n\text{ is 0}\\
                             1~~~\text{if }n\text{ is 1}\\
                             0~~~\text{otherwise}
                           \end{cases}
  \\
  F_3(n) = F(F(F(F_0(n)))) & = \begin{cases}
                                1~~~\text{if }n\text{ is 0}\\
                                1~~~\text{if }n\text{ is 1}\\
                                2~~~\text{if }n\text{ is 2}\\
                                0~~~\text{otherwise}
                              \end{cases}
\end{align*}

In this case it is easy to see that $F_0 \leq F_1 \leq F_2 \leq F_3 \leq \dots$. In
the general case this follows from the fact that $F$ has to be monotone and $F_0$ is
always the least element of the function space $F$ operates on. 


\subsubsection{Interpreting random definitions}\label{sec:random-interp}
% Describe how we use omega-cpos in order to express the probabilistic interpretation
% of our programs in pwhile and rml.

\todo[inline, color=green!40]{A measure is a linear function $\mu$ from a set A to
  non-negative real numbers. It also preserves least upper bounds. \\

Mß = (ß$\to$[0,1])$\to$[0,1]
The (ß$\to$[0,1]) part describes a probability distribution. 
We can view our programs as transformations of probability distributions.\\

"a term e of type ß is translated to a purely functional one [e] which is
understood as a measure on the same type."}

\todo[inline]{The w-cpo structure on [0,1]; why is it an w-cpo?\\
@Bas: What do we need the structure for? \\
I am confused by what it says on page 574: First we talk about $\mu$(f) for f:A$\to$[0,1]
and then we say that $\mu$ is a measure on A. Shouldn't it be on probability
distributions over A (so A$\to$[0,1])?}

\todo[inline]{Monadic transformation. Should we add a subsection on monads or should we just
  mention them?}

% Packages: comment


\section{Our approach}
% Goal: 6-12 pages

% What we tried to do and the theory behind.

% Describe the translation from while to functional (+recursion), to lambda-calculus,
% to an interpretation in omega-cpos.

% We planned on making an interpretation of Rml in Coq and then translating the xhl
% implementation of pwhile to our abstract syntax. 


\subsection{Translating \texttt{while} to a functional language}
% 
In order to do the translations properly, let us first have a look at a translation
from the simple, widely known \texttt{while} language to a simple functional language
resembling \rml. The thought behind this is that once this translation is in place,
all we have to do to translate \texttt{pwhile} to \rml is to add probability.

\begin{align}
  \label{eq:while}
  exp~  ::=~~ & x \vert n \vert \texttt{true} \vert \texttt{false} \vert f~x \\
  stm~  ::=~~ & \texttt{skip} \vert x := e
               \vert \texttt{if } e \texttt{ then } s_1 \texttt{ else } s_2
               \vert \texttt{while } e \texttt{ do } s \vert s_1;s_2
\end{align}

The syntax of our functional language is the same as \rml modulo the pre-defined
randomised functions.

The translation of expressions is completely straightforward: variables are mapped to
variables, constants to constants, and function applications to function applications. 

In order to translate statements we choose a set of SML-style matching rules; this
choice is due to the translation of sequences being dependent on what the first
statement is. We will in the following write the translation of a \texttt{while}
statement $s$ to an expression in our functional language as $\llbracket s
\rrbracket$. 

The result of a computation in \texttt{while} is the state of a memory, while the
result of a functional computation is a value. A simple way to make up for this
difference is by choosing a variable name that is designated the return variable and
encapsulates the information we are interested in after the computation. This is the
result of a program translated from \texttt{while} to our functional language; in the
following we choose $x_r$ as the symbol for the chosen return variable. 
\todo{@Bas: Is this description of our return and $x_r$ sufficiently clear?}

\begin{align}
  \label{eq:transwhile}
  \texttt{skip}~;~s & ~\mapsto~ & \llbracket s \rrbracket\\
  \texttt{skip}     & ~\mapsto~ & x_r\\
  x := e~;~s        & ~\mapsto~ & \texttt{let } x := e \texttt{ in }\llbracket s \rrbracket\\
  x_r := e          & ~\mapsto~ & e \\
  x := e            & ~\mapsto~ & x_r \\
  (\texttt{if } e \texttt{ then } s_1 \texttt{ else } s_2) ~;~s_3
                    & ~\mapsto~ & \texttt{if } e
                                  \texttt{ then } \llbracket s_1 ~;~ s_3 \rrbracket
                                  \texttt{ else } \llbracket s_2 ~;~ s_3 \rrbracket \\
  \texttt{if } e \texttt{ then } s_1 \texttt{ else } s_2
                    & ~\mapsto~ & \texttt{if } e
                                  \texttt{ then } \llbracket s_1 \rrbracket
                                  \texttt{ else } \llbracket s_2 \rrbracket \\
  (\texttt{while } e \texttt{ do } s_1) ~;~ s_2
                    & ~\mapsto~ & \texttt{let rec } f~x :=
                                  \texttt{if } e
                                  \texttt{ then } \llbracket s_1 ~;~ f~x \rrbracket
                                  \texttt{ else } \llbracket s_2 \rrbracket \label{eq:while1}
                                  \texttt{ in } f~0\\
  \texttt{while } e \texttt{ do } s_1
                    & ~\mapsto~ & \texttt{let rec } f~x :=
                                  \texttt{if } e
                                  \texttt{ then } \llbracket s_1 ~;~ f~x \rrbracket
                                  \texttt{ else } x_r \label{eq:while2}
                                  \texttt{ in } f~0
\end{align}

Note that in \ref{eq:while1} and \ref{eq:while2} we create recursive functions with a
name and an argument, both of which are not present in the while construct we
translate from. This means that we have to be careful about the translation: Both $f$
and $x$ have to be chosen fresh; and even fresher than that, they can not occur in
the body of the while loop we are translating either, because that would break the
recursive call.

Further notice that the recursive functions are always called with a dummy
argument. This is because they act as procedures, but since our syntax requires an
argument for recursive definitions, we give a dummy argument.


\subsection{Translation from Rml to typed $\mathbf{\lambda}$-calculus}
\todo[inline, color=red]{This section is preliminary and needs either huge changes or deletion
  before the report is finalised. }
\begin{center}
  \begin{tabular}{|c|c|}
    \hline
    Rml & typed $\lambda$-calculus \\ \hline
    Var $(x,A)$ & $x : A$ \\
    Const $A$ $c$ & $c : A$ \\
    Let $(x,A)$ $e_1$ $e_2$ & $(\lambda x : A. e_2)~e_1$ \\
    Fun $(x,A)$ $e$ & $\lambda x : A. e$ \\
    App $e_1$ $e_2$ & $e_1~e_2$ \\
    Let rec $(f,A \rightarrow B)$ $(x,A)$ $e_1$ $e_2$ & $(\lambda f : A \rightarrow B. e_2)~(Y~(\lambda f : A \rightarrow B. \lambda x : A. e_1))$ \\ \hline
  \end{tabular}
\end{center}
The problem here is that we need to translate \(e1\) and \(e2\) to their simple forms, so we do an intermediate translation:
\[Let\]

\subsubsection{Example: Fib}
Expression:
\begin{align*}
\mathtt{Let\_rec}&~(f,\mathbb{N} \rightarrow \mathbb{N})~(x,\mathbb{N})~\\
  (&\mathtt{if}~x \leq 0\\
                 &\mathtt{then}~0\\
                 &\mathtt{else}~f~(x - 1) + f~(x-2))\\
  (&f~3)
\end{align*}
Typing:
\begin{align*}
\mathtt{Let\_rec}&~(f,\mathbb{N} \rightarrow \mathbb{N})~(x,\mathbb{N})~\\
  ((&\mathtt{if}~(x \leq 0 : \mathbb{B})\\
                 &\mathtt{then}~(0 : \mathbb{N})\\
                 &\mathtt{else}~(f : \mathbb{N} \rightarrow \mathbb{N})~(x - 1 : \mathbb{N}) + (f : \mathbb{N} \rightarrow \mathbb{N})~(x-2 : \mathbb{N}) : \mathbb{N}) : \mathbb{N})\\
  (&(f : \mathbb{N} \rightarrow \mathbb{N})~(3 : \mathbb{N}) : \mathbb{N})
\end{align*}
Semi-simple
\begin{align*}
  &\mathtt{Let\_stm}~f \\
  &\quad \mathtt{sFix} \\
  &\quad \quad \mathtt{sFun}~(f,\mathbb{N} \rightarrow \mathbb{N})\\
  &\quad \quad \quad \mathtt{sFun}~(x,\mathbb{N}) \\
  &\quad \quad \quad \quad ((\mathtt{if}~(x \leq 0)\\
  &\quad \quad \quad \quad ~~\mathtt{then}~0\\
  &\quad \quad \quad \quad ~~\mathtt{else}~f~(x - 1) + f~(x-2)))\\
  &\quad (f~3)
\end{align*}
Simple form:
\begin{align*}
  &\mathtt{sApp}~\mathtt{sFix} \\
  &\quad \mathtt{sFun}~(f,\mathbb{N} \rightarrow \mathbb{N})\\
  &\quad \quad \mathtt{sFun}~(x,\mathbb{N}) \\
  &\quad \quad \quad ((\mathtt{if}~(x \leq 0)\\
  &\quad \quad \quad ~~\mathtt{then}~0\\
  &\quad \quad \quad ~~\mathtt{else}~f~(x - 1) + f~(x-2))) \\
  &\quad 3
\end{align*}


\subsection{Interpreting $\lambda$-calculus in the space of $\omega$-cpos}
\todo[inline]{What do $\omega$-cpos have to do with this?}


\subsection{Interpreting \texttt{while} directly}
\todo[inline]{This should probably mainly refer back to the interpretation of
  \texttt{pwhile}.}


\subsection{All translations (forward)}
\todo[inline]{What is the point of this section? }
\begin{center}
  \rotatebox{0}{
    \resizebox{1.05\linewidth}{!}{
      \begin{tabular}{|c|c|c|}
        \hline
        Rml & @sRml A & typed $\lambda$-calculus \\ \hline
        Var $(x,A)$ & sVar $x$ & $x : A$ \\
        Const $A$ $c$ & sConst c & $c : A$ \\
        Let $(x,T)$ $e_1$ $e_2$ & $e_2'$ & $(\lambda x : T, e_2 : A)~(e_1 : T) : A$ \\
        Fun $(x,T)$ $e$ & sFun $S$ $(x,T)$ $e'$ & $(\lambda x : T, e : S) : T \rightarrow S$ \\
        App $T$ $e_1$ $e_2$ & sApp $T$ $e_1'$ $e_2'$ & $(e_1 : T \rightarrow A)~(e_2 : T) : A$ \\ && \\
        Let rec $T$ $S$ $f$ $x$ $e_1$ $e_2$ &
                                              \begin{tabular}{l}
                                                sApp $(T \rightarrow S)$ \\
                                                \quad (sFun $A$ $(f,T \rightarrow S)$ $e_2'$) \\
                                                \quad (sFun $S$ $(x,T)$\\
                                                \quad \quad (sFix $T$ $f$ $x$ $e_1'$ (sVar $x$)))
                                              \end{tabular}
&
                                                              \begin{tabular}{l}
                                                                $(\lambda f : T \rightarrow S, e_2 : A)$\\
                                                                \quad $(Y~(\lambda f : T \rightarrow S, \lambda x : T, e_1 : S) : T \rightarrow S) : A$
                                                              \end{tabular}
\\ \hline
      \end{tabular}
    }}
\end{center}





\section{Our contribution}
% Goal: 6-12 pages
% What we actually did.

% We implemented an abstract syntax for Rml in Coq

% We tried to translate from pwhile to said Rml (syntax to syntax)

% We were able to interpret Rml modulo general recursion, using a probability monad.

% We tried implementing general recursion with the extra step of translating to the
% typed lambda calculus and using a fixpoint combinator.

\section{Comparisons and future work}
% Goal: 2-4 pages
% What the section header says


\section{Conclusion}
% Goal: 1-2 pages



\newpage
\section{Appendix}
Example - Error: Stack Overflow.
\begin{lstlisting}[language=coq]
Fixpoint replace_all_variables_aux_type
         A (x : Rml) (env : seq (nat * Type * Rml))
         (fl : seq (nat * Type)) `{env_valid : valid_env env fl}
         `{x_valid : @rml_valid_type A (map fst env) fl x} : @sRml A
                                                                            
with replace_all_variables_aux_type_const
       A0 A a (env : seq (nat * Type * Rml))
       (fl : seq (nat * Type)) `{env_valid : valid_env env fl}
       `{x_valid : @rml_valid_type A0 (map fst env) fl (Const A a)} : @sRml A0
with replace_all_variables_aux_type_let
       A p x1 x2 (env : seq (nat * Type * Rml))
       (fl : seq (nat * Type)) `{env_valid : valid_env env fl}
       `{x_valid : @rml_valid_type A (map fst env) fl (Let_stm p x1 x2)} : @sRml A
with replace_all_variables_aux_type_fun
       A T p x (env : seq (nat * Type * Rml))
       (fl : seq (nat * Type)) `{env_valid : valid_env env fl}
       `{x_valid : @rml_valid_type A (map fst env) fl (Fun_stm T p x)} : @sRml A
with replace_all_variables_aux_type_if
       A x1 x2 x3 (env : seq (nat * Type * Rml))
       (fl : seq (nat * Type)) `{env_valid : valid_env env fl}
       `{x_valid : @rml_valid_type A (map fst env) fl (If_stm x1 x2 x3)} : @sRml A
with replace_all_variables_aux_type_app
       A T x1 x2 (env : seq (nat * Type * Rml))
       (fl : seq (nat * Type)) `{env_valid : valid_env env fl}
       `{x_valid : @rml_valid_type A (map fst env) fl (App_stm T x1 x2)} : @sRml A
with replace_all_variables_aux_type_let_rec A T T0 n n0 x1 x2 (env : seq (nat * Type * Rml))
     (fl : seq (nat * Type)) `{env_valid : valid_env env fl}
     `{x_valid : @rml_valid_type A (map fst env) fl (Let_rec T T0 n n0 x1 x2)} : @sRml A.
Proof.
  (** Structure **)
  {
    induction x ; intros ; refine (sVar (0,A)).
  }
  
  all: refine (sVar (0,A)).  
Defined.
\end{lstlisting}

\end{document}
